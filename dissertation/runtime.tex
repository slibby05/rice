\documentclass{article}
\usepackage{qtree}

\begin{document}
Recall that Curry is fundamentally a graph rewriting system.
Curry programs are compiled into rewriting rules,
and the execution of a curry program applies these rules to rewrite a graph called \texttt{main}.
This isn't a new concept.  All lazy functional languages are fundamentally graph rewriting systems.
However, Curry differs from other systems in one important respect.
Curry allows for non-determinism, which means that most Curry programs are non-confluent rewrite system.
This has far reaching consequences throughout the runtime system of Curry.

If we want to design a runtime system for Curry we really need to design a system for
representing and rewriting graphs.
We have a few questions to answer.
How do we represent the graph?
How do we update nodes in the graph?
How do we deal with choice nodes?
How do we deal with free variable nodes?

\subsection{Representing the Graph}

The first question is initially pretty straightforward.
Although, we will have to modify our answer later as we add more features.
A rewrite graph is simply a directed rooted graph.
These have a very simple implementation in computer systems.
Each node in the graph contains pointers to its child nodes.
We can represent this with the following C struct:

\begin{verbatim}
typedef struct Node
{
    unsigned int tag;
    unsigned int arity;
    char* name;
    void (*hnf)(struct Node*);
    struct Node* children[4];
} Node;
\end{verbatim}

Here the \texttt{name}, and the \texttt{arity} fields aren't strictly necessary,
but they are useful for debugging.
The \texttt{tag} field determines what type of node we are looking at,
with options for \texttt{FUNCTION} or one of the possible constructors.
The \texttt{hnf} field is a pointer to a function that can reduce the current node to head normal form.
Note that if \texttt{tag} is a \texttt{FUNCTION}, then the \texttt{hnf} field can be set to null.
Finally the \texttt{children} field contains an array of 3 children for the node.
If the node needs more than 3 children, then we can extend the array as needed.
This is accomplished by adding a pointer to an array of nodes to the 4th slot in \texttt{children}

It should be noted that since \texttt{tag}, \texttt{arity}, \texttt{name}, and \texttt{hnf}
are constant for a given node, we could put all of these fields in a symbol object.
This has the advantage of being more efficient to update, but it also requires two memory accesses
to call the \texttt{hnf} function, and \texttt{tag} field.

\subsection{Reducing the Graph}

Initially reducing the graph is really simple.
In order to reduce a graph to normal form, we reduce the root of the graph to head normal form.
Then we reduce each child to normal form.

This can be done with the following algorithm:
\begin{verbatim}
void nf(Node* expr)
{
    if(expr->hnf)
    {
        expr->hnf(expr);
    }
    for(int i = expr->arity-1; i >= 0; i--)
    {
        nf(expr->children[i]);
    }
}
\end{verbatim}

This algorithm is simple and efficient.
However, we will have to change it slightly when we add in more features

While this algorithm is good for computing normal forms, it doesn't
tell us how to compute head normal forms.
To compute the head normal forms, we need to compile the curry code.
Our goal is to generate one C function for every Curry function, after transformations and optimizations have been applied.
The C function will take a Node repressing the Curry function.
It will then modify the subgraph rooted by that node with the result of the corresponding rewrite rule..
For example the expression:
\begin{verbatim}
[1,2] ++ [3,4]
\end{verbatim}
\Tree[.\texttt{++} [.\texttt{:} \texttt{1} [.\texttt{:} \texttt{2} \texttt{[]} ] ]
                   [.\texttt{:} \texttt{3} [.\texttt{4} \texttt{2} \texttt{[]} ] ] ]
                     
is rewritten to
\begin{verbatim}
1 : (2 ++ [3,4])
\end{verbatim}
\Tree[.\texttt{:} \texttt{1} 
                   [.\texttt{++} [.\texttt{:} \texttt{2} \texttt{[]} ]
                   [.\texttt{:} \texttt{3} [.\texttt{4} \texttt{2} \texttt{[]} ] ] ] ]

We will give a full translation from Curry code to C code later,
but for now, the idea can be best understood with an example.

Let's look at the code for \texttt{++}
\begin{verbatim}
[] ++ ys = ys
(x:xs) ++ ys = x : (xs++ys)
\end{verbatim}

Which is given by the definitional tree\\
\Tree[.\texttt{xs++ys} [.\texttt{[]} \texttt{ys} ] [.\texttt{(z:zs)} \texttt{z : (zs++ys)} ] ]\\

Our translation is going to produce C code to execute the operation represented by the definitional tree.
The idea is straightforward.  For every branch in the definitional tree we have a case.
We then execute the matching branch.
For out example of \texttt{++}, the root of our expression has two children \texttt{xs} and \texttt{ys}.
So we give those children names.

Now the tree branches on \texttt{xs}, so we need to determine if \texttt{xs} is a function,
or if it's a constructor, we need to determine which constructor it is.
This is the purpose of the tag field.
Each type has a set of tags that are integers between 0 and $c$, where $c$ is the number of constructors.
The value 0 will represent that the node is a function, while 1 will represent the first constructor.
We will need to update this later as we consider more complicated expressions, but for now this will work.

So, we can represent the branching with a \texttt{switch} statement in C.
We switch on the variable \texttt{xs}, and take the appropriate branch.
If the branch is a \texttt{FUNCTION}, then we reduce the node and try again.

Finally, we translate all names to be legal C names, so \texttt{++} becomes \texttt{\_PL\_PL}.
We've also created a set of functions \texttt{set\_NAME} and \texttt{make\_name}.
for each function and constructor symbol, the first function will set a node to that type,
and the second function will allocate a new node of that type.
So \texttt{set\_\_PL\_PL(root, ...)} will set the root node to be a \texttt{++} node.
These functions are purely for readablilty, and are inlined in the actual compiler.
The function \texttt{set\_node(x, y)} will set node \texttt{x} to be a copy of \texttt{y}.
We can see a simple version of code to reduce an append function below.

\begin{verbatim}
void _PL_PL_hnf(Node* root)
    Node* xs = root->children[1];
    Node* ys = root->children[0];

    while(true)
    {
        switch(xs->tag)
        {
            case FUNCTION:
                xs->hnf(xs);
                break;

            case _LB_RB: // [] for nil node
                set_node(root, ys);
                return;

            case _CO: // : for cons node
                Node* z = xs->children[1];
                Node* zs = xs->children[0];
                set_node(root, make__CO(z, make__PL_PL(zs,ys)));
                return;
        }
    }
}
\end{verbatim}

This code will reduce the \texttt{xs} node until it is in head normal form.
Then \texttt{xs} must be rooted by a constructor.
At which point, one of the second two cases will run.
This will either return the node \texttt{ys} if \texttt{xs} is nil,
or it will return \texttt{z : zs++ys} if \texttt{xs} isn't nil.
This is exactly the behavior we want from append, but there are a couple of caveats.

First, we access the children of a node in reverse.
It turns out to be more convenient to store the children in reverse,
so we adopt that convention here.

Next, and more importantly, this trick with the while loop won't work later.
It's also inefficient.
If the variable in the case is rooted by a function, then we need to reduce it with a call to hnf,
finish the loop, jump back to the top, then jump to the next case.
It would be better if we could jump to the next case directly.
So, instead of trying to make the C-style switch statement work,
we're going to borrow a trick from bytecode interpreters, and construct a jump table.
Since we know the values of all of the possible branches, we can make them contiguous.
\texttt{FUNCTION} has value 0, \texttt{\_LB\_RB} has value 1 and \texttt{\_CO} has value 2.
Now we can make a table, and jump to the correct value.

Finally, there is one serious, but subtle, error with the above code.
If \texttt{xs} is nil, then we set the root to be a copy of \texttt{ys},
but not the node itself.
This would be correct in a functional language, 
but this will cause problems when we introduce non-determinism.
So, we must introduce a new type of node.
A \texttt{FORWARD} node is simple a node that points to another node.
We will give the \texttt{FORWARD} tag a value of 1, and bump all of the constructor nodes up accordingly.
This results in the new code for \texttt{\_PL\_PL\_hnf}.

\begin{verbatim}
void _PL_PL_hnf(Node* root)
{
    Node* xs = root->children[1];
    Node* ys = root->children[0];

    static void* table[] = {&&FUNCTION, 
                            &&FORWARD, 
                            &&_LB_RB, 
                            &&_CO};

    goto *table[xs->tag];

    FUN:
    xs->hnf(xs);
    goto *table[xs->tag];

    FORWARD:
    while(xs->children[0]->tag == FORWARD_TAG)
        xs->children[0] = xs->children[0]->children[0];
    xs = xs->children[0];
    goto *table[xs->tag];

    _LB_RB:
    {
        if(ys->tag == FUNCTION_TAG)
        {
            ys->hnf(ys);
        }
        set_forward(root, ys);
        return;
    }

    _CO:
    {
        Node* z  = xs->children[1];
        Node* zs = xs->children[0];
        set_CONS(root, z, make__PL_PL(zs, ys));
        return;
    }
}
\end{verbatim}

Most of the code is similar,
however the \texttt{FORWARD} and \texttt{\_LB\_RB} cases require some explanation.
First the forward case must traverse a chain of nodes until it encounters a node that isn't a forward node.
It is possible to have a graph with a long chain of forward nodes.
While we are walking this chain we might as well do path compression.
It costs almost nothing, and prevents us from having to walk the chain again.

The \texttt{\_LB\_RB} case sets the root node to a forward node, 
but first we compute the head normal form of \texttt{ys}.
This step is important, because the \texttt{hnf} functions guarantee
that the root will be in a head normal form after they're called.
If we don't reduce \texttt{ys}, then it's possible that it won't be in head normal form, 
and now the root is pointing to an Unreduced function.

This scheme works fine, but we're still missing higher order functions, and non-determinism.
Next will add in higher order functions, and partial application.

\subsection{Higher order functions, and partial application}

If we're only applying known functions to the correct number of arguments, then function application is easy.
However, this isn't normally the case with functional languages.
We may not know what the function is, such as in \texttt{map}.
We may also have applied the function to too few arguments, or too many arguments.

We need to deal with both of these cases.
The modern approach is defunctionalization.
Instead of trying to deal with higher order functions directly, we create an \texttt{apply} node.
This node is distinct because it has a variable number of children.
An apply node will always have a function \texttt{f}, a number \texttt{n}, 
but it will also have a list of arguments to apply \texttt{f} to \texttt{a1,a2...an}.
The layout of this node will also be deferent.
\texttt{f} will always be in position 0, and \texttt{n} will be in position 1.
However, the rest of the arguments will be in the extension array for node children.
That make it easier to copy arguments over if they're all in the same array.

The only valid constructor for \texttt{apply} is a partial application node.
These nodes will only ever have two arguments.
The first argument is the function that is being applied,
and the second argument is the number of missing children before it can be applied.

The resulting code for dealing with an apply node is pretty straightforward.
We simply copy over children from the apply node to the partial application until we 
fully saturate the partial application, or we run out of arguments in the apply node.
If we've saturated the partial application, then we reduce it an apply more arguments.
If we ran out of arguments in the apply node, then we return a new partial application node.
This can be implemented with the following code.


\begin{verbatim}
void apply_hnf(Node* root)
{
    Node* part = root->children[0];
    int n      = (int)root->children[1];

    static void* table[] = {&&FUNCTION, 
                            &&FORWARD, 
                            &&PARTAPP,

    goto *table[part->tag];

    FUN:
    part->hnf(part);
    goto *table[part->tag];

    FORWARD:
    while(part->children[0]->tag == FORWARD_TAG)
        part->children[0] = part->children[0]->children[0];
    part = part->children[0];
    goto *table[part->tag];

    PARTAPP:
    {
        Node* f = copy_node(part->children[0]);
        int missing = part->children[1];
        while(missing > 0 && n > 0)
        {
            f->children[missing-1] = root->children[3][n-1];
            n--;
            missing--;
        }
        if(missing > 0)
        {
            set_partapp(root, f, missing);
            return;
        }
        if(n > 0)
        {
            f->hnf(f);
            root->children[0] = f;
            root->children[1] = (Node*) n;
            apply_hnf(root);
            return;
        }
    }
}
\end{verbatim}

In order to incorporate higher order functions, such as \texttt{map},
we simply return an \texttt{apply} node whenever we use the higher order function.

Now that we have higher order functions, we have implemented a lazy functional language.
However, we still need to add non-determinism and free variables.

\subsection{Non-determinism}

In order to incorporate non-determinism we will need to make a more substantial change
than the one for higher order functions.
First we need a strategy for computing all solution to a non-deterministic functions.
While there are a few strategies out there,
there is no implementation of a complete strategy for Curry.
Because of this, and because the purpose of this dissertation is to investigate optimizations,
I will be implementing a backtracking strategy.
It's easier to implement, and it's more efficient that pull tabbing or bubbling.

To implement non-determinism we need to introduce two new types of nodes.
A \texttt{CHOICE} node has two children.  This node will introduce non-determinism.
A \texttt{FAIL} node has no children, and represents a failed computation.
This will remove non-determinism from the program.
This means that we now have five types of tags, FAIL, FORWARD, FUNCTION, CHOICE, and constructor tags.

Our backtracking strategy will rely on the ability to ``undo'' computations.
In order to get this to work, we need to store the computations that we've done.
We will use a frame to do this.
A frame is a simple object.
It contains a node that has been rewritten, and a copy of the original node.
We can then backtrack by keeping a stack of frames.
When we do a rewrite step, we put a frame with the current node, and a copy of the original on the stack.
When we need to backtrack, we pop frames off the stack, and replace the current node with the original one.
In fact, the backtracking algorithm is quite short.

\begin{verbatim}
void undo()
{
    if(bt_stack->size == 0)
        return;

    NodePair* frame = pop(bt_stack);

    while(!frame->choice)
    {
        memcpy(frame->lhs, frame->rhs, sizeof(Node));

        frame = pop(bt_stack);
    }
    memcpy(frame->lhs, frame->rhs, sizeof(Node));
}
\end{verbatim}

No we just need to add code for pushing nodes on the stack
and dealing with choices.
The code doesn't actually change too much from the previous iteration.
We need to add the \texttt{FAIL}, and \texttt{CHOICE} branches to our case,
but those are the only major changes.
The fail branch is pretty simple, if a needed Subexpression failed, then the entire expression fails.
So, we can just set the entire expression to be a fail node.

The choice branch is a little more complicated.
We want to overwrite the root with the left choice,
but we need to keep the right choice around.
Then we push a new frame on the stack.
The final optional parameter to the \texttt{push} function
indicates that this is a choice frame for the \texttt{undo} function.

Finally, before we replace the root, we push a frame with the root onto the stack
so that this rewrite can be undone later.

\begin{verbatim}
void append_hnf(Node* root)
{
    Node* xs = root->children[1];
    Node* ys = root->children[0];

    static void* table[] = {&&FAIL, 
                            &&FUN, 
                            &&FORWARD, 
                            &&CHOICE, 
                            &&_BL_BR, 
                            &&_CO};
    goto* table[xs->tag];

    FAIL:
    push(bt_stack, root, copy_node(root));
    set_fail(root);
    return;

    FUN:
    xs->hnf(xs);
    goto* table[xs->tag];

    FORWARD:
    while(xs->children[0]->tag == FORWARD_TAG)
        xs->children[0] = xs->children[0]->children[0];
    xs = xs->children[0];
    goto* table[xs->tag];

    CHOICE:
    Node* left  = root->children[0];
    Node* right = root->children[1];
    left->hnf(left);
    memcpy(root, left, sizeof(Node));
    stack_push(bt_stack, root, right, true);
    goto* table[xs->tag];

    NIL:
    {
        push(bt_stack, root, copy_node(root));
        if(FUNCTION_TAG <= ys->tag &&
           ys->tag <= CHOICE_TAG)
        {
            ys->hnf(ys);
        }
        set_forward(root, ys);
        return;
    }

    CONS:
    {
        push(bt_stack, root, copy_node(root));

        Node* z = xs->children[1];
        Node* zs = xs->children[0];

        set__CO(root, z, make__PL_PL(zs, ys));
        return;
    }
}
\end{verbatim}

Now that we have Non-determinism, it's a simple step to add free variables.

\subsection{Free variables}

Free variables are computed through narrowing.
This means that they only get a value when they're the argument in a case.
So, we can fill in the value of a free variable by adding a branch to the case.
We'll initially set the free variable to be the first constructor,
and then we'll add a choice for every other constructor of that type.
If the constructor has any children, we set those to be free variables.

\begin{verbatim}
void append_hnf(Node* root)
{
    Node* xs = root->children[1];
    Node* ys = root->children[0];

    static void* table[] = {&&FAIL, 
                            &&FUN, 
                            &&FORWARD, 
                            &&CHOICE, 
                            &&FREE,
                            &&_BL_BR, 
                            &&_CO};
    goto* table[xs->tag];

    FAIL:
    push(bt_stack, root, copy_node(root));
    set_fail(root);
    return;

    FUN:
    xs->hnf(xs);
    goto* table[xs->tag];

    FORWARD:
    while(xs->children[0]->tag == FORWARD_TAG)
        xs->children[0] = xs->children[0]->children[0];
    xs = xs->children[0];
    goto* table[xs->tag];

    CHOICE:
    Node* left  = root->children[0];
    Node* right = root->children[1];
    left->hnf(left);
    memcpy(root, left, sizeof(Node));
    stack_push(bt_stack, root, right, true);
    goto* table[xs->tag];

    FREE:
    push_choice(v1, make_CONS(free_var(), free_var()));
    stack_push(bt_stack, xs, make_CONS(free_var(), free_var()), true);
    set_NIL(xs);
    goto* table[xs->tag];

    NIL:
    {
        push(bt_stack, root, copy_node(root));
        if(FUNCTION_TAG <= ys->tag &&
           ys->tag <= CHOICE_TAG)
        {
            ys->hnf(ys);
        }
        set_forward(root, ys);
        return;
    }

    CONS:
    {
        push(bt_stack, root, copy_node(root));

        Node* z = xs->children[1];
        Node* zs = xs->children[0];

        set__CO(root, z, make__PL_PL(zs, ys));
        return;
    }
}
\end{verbatim}

This is the complete code for implementing \texttt{++}.
While this code will work, we can actually make the non-determinism more efficient.

\subsection{Efficient Non-determinism}

The problem with the current implementation of non-determinism is that recompute several expressions that we don't need to.
Consider the expression \texttt{(1 + 2 + 3) + (4 ? 5)}.
This expression will compute \texttt{(1 + 2 + 3)} twice, but it really shouldn't.
The subexpression is deterministic even if the entire expression isn't.
It would be nice if we could avoid this.

One possible way to avoid these computations is to mark every expression with a flag.
This flag says whether the expression was deterministic after it was evaluated.
If an expression is deterministic then we don't need to push a frame on the backtracking stack for it.

This is great, but in order for this to work we need to know when an expression was non-deterministic.
Fortunately this is really easy to check.\\
$\ $\\
\textbf{theorem:}
For any expression $e$, $e$ is deterministic if, and only if, all needed subexpressions of $e$ are deterministic.\\
$\ $\\
Proof:\\
The forward direction is clearly true, if $e$ is non-deterministic, then it must have taken a non-deterministic step.
But this can only happen if it had a non-deterministic needed subexpression.
The reverse direction follows by induction on the reduction of $e$.
If $n$ is needed in $e$, then $n$ can't take a non-deterministic step, or $e$ would be non-deterministic.
By the inductive hypothesis any needed node for $n$ must also be deterministic.

This gives us a simple algorithm for determining is an expression is deterministic.
We simply percolate up determinism information as we're evaluating the expression.
If we ever have a case on a non-deterministic subexpression, then we are also non-deterministic.

Fortunately this extension doesn't add must to our generated code.
We only push the root on the backtracking stack if the argument to the case was non-deterministic.
We also have a small optimization to creating the forward node.
It turns out that if \texttt{ys} was deterministic, then it's perfectly valid to copy the
contents of \texttt{ys} into the root.
This saves us having to construct a forward node.

\begin{verbatim}

void append_hnf(Node* root)
{
    Node* xs = root->children[1];
    Node* ys = root->children[0];

    static void* table[] = {&&FAIL, 
                            &&FUN, 
                            &&FORWARD, 
                            &&CHOICE, 
                            &&FREE, 
                            &&NIL, 
                            &&CONS};
    goto* table[xs->tag];

    FAIL:
    if(xs->nondet)
    {
        push(bt_stack, root, copy_node(root));
        root->nondet = true;
    }
    fail(root);
    return;

    FUN:
    xs->hnf(xs);
    goto* table[xs->tag];

    FORWARD:
    while(xs->children[0]->tag == FORWARD_TAG)
        xs->children[0] = xs->children[0]->children[0];
    xs = xs->children[0];
    goto* table[xs->tag];

    CHOICE:
    Node* left  = root->children[0];
    Node* right = root->children[1];
    left->hnf(left);
    memcpy(root, left, sizeof(Node));

    root->nondet = true;
    stack_push(bt_stack, root, right, true);
    goto* table[xs->tag];

    FREE:
    stack_push(bt_stack, xs, make_CONS(free_var(), free_var()), true);
    set_NIL(xs);
    xs->nondet = true;
    goto* table[xs->tag];

    NIL:
    {
        if(xs->nondet)
        {
            push(bt_stack, root, copy_node(root));
            root->nondet = true;
        }
        if(FUNCTION_TAG <= ys->tag &&
           ys->tag <= CHOICE_TAG)
        {
            ys->hnf(ys);
        }
        if(ys->nondet)
        {
            push(bt_stack, root, copy_node(root));
            root->nondet = true;
            forward(root, ys);
        }
        else
        {
            set_node(root, ys);
        }
        return;
    }

    CONS:
    {
        if(xs->nondet)
        {
            push(bt_stack, root, copy_node(root));
            root->nondet = true;
        }
        Node* z = xs->children[1];
        Node* zs = xs->children[0];
        set__CO(root, z, make__PL_PL(zs, ys));
        return;
    }
}
\end{verbatim}
\end{document}
