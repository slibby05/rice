
There is a problem with local variables.
Consider the function

\begin{verbatim}
main = let x = True ? False
       in case x of
               True -> 1
               False -> 0
\end{verbatim}

When we go to evaluate this function we would expect to produce two answers, 0 and 1.
Unfortunately that's not what happens.
We'll generate the following code.
\begin{verbatim}
main_hnf(Node* root)
    Node* x = make_choice(make_True(), make_False());
    static void* table[] = {&&FAIL, &&FUNCTION, &&FORWARD, 
                            &&CHOICE, &&FREE, &&TRUE, &&FALSE};
    
    goto* table[x->symbol->tag];
  
    FAIL:
    {
        if(x->nondet)
        {
            save(root);
        }
        fail(root);
        return;
    }
 
    ...
  
    CHOICE:
    {
        choose(x);
        goto* table[x->symbol->tag];
    }
    ...
  
    TRUE:
    {

        if(x->nondet)
        {
            save(root);
        }
        set_1(root);
        return;
    }
    FALSE:
    {

        if(x->nondet)
        {
            save(root);
        }
        set_0(root);
        return;
    }
}
\end{verbatim}

First we create the local variable \texttt{x = ?(True,False)}.
Then we evaluate \texttt{x} to be \texttt{True}, and match the first case, 
push \texttt{root} on the backtracking stack, and reduce \texttt{root} to \texttt 1.
So for this is fine, but what happens when we backtrack?
Our stack contains two rules \texttt{[1 -> main, x -> ?(0, 1)}.
Notice that the variable we made a choice on, \texttt x, isn't mentioned in the first rule.

This is where the problem arises.  We undo our rewrite, and replace the node \texttt 1 with \texttt{main}.
Now when we evaluate \texttt{main} we follow the same proceedure,
but this time we create a new local variable \texttt{x}.
This will again cause \texttt{main} to reduce to \texttt 1.
This process will repreat, continually growing our backtracking stack.

At this point we could reasonably ask how this problem could arise?
The semantics of Curry, and graph rewriting in general, have already been established.
Surely, this problem must be accounted for.
The answer is surprisingly simple, and completely worthless to this project.
Graph rewriting doesn't allow for local variables, so this issue can't crop up.
Furthermore, prior implementations of Curry avoid this issue by transforming their
source programs into graph rewrite systems.

The above program would become:
\begin{verbatim}
main = let x = True ? False
       in main1 x

main1 x = case x of
               True  -> 1
               False -> 0
\end{verbatim}

This transformation is consistent, easy to implement, complete, and innefficient.
Furthermore, this explicitly undoes a lot of the work accomplished by the optimizations.
So, we need a new way to handle local variables.

While this solution isn't viable for this compiler, maybe we can play around with this idea.
The problem we have is that we may decide to backtrack at several different points in the function.
Really, we need a way to save the current function we're reducing, along with it's current state.
One possible way would be to store the function we're reducing, all currently defined local variables,
and a pointer to the position in code where we pushed this rule on the stack.
This would store all of the relavant information, but it is also difficult to understand and implement correctly.

Fortunately there's a simpler solution.
Instead of compiling a single function, we compile a set of functions.
This is easiest to see with an example.

Consider the function choose3:
\begin{verbatim}
choose3 = let x = 0 ? 1
          in case x of 
                  0 -> let y = 0 ? 1
                  in case y of
                          0 -> let z = 0 ? 1
                               in case z of
                                       0 -> 0
                                       1 -> 1
                          1 -> 2
                  1 -> 3
\end{verbatim}

We will turn this into 3 different function.

\begin{verbatim}
choose3 = let x = 0 ? 1
          in case x of 
                  0 -> let y = 0 ? 1
                  in case y of
                          0 -> let z = 0 ? 1
                               in case z of
                                       0 -> 0
                                       1 -> 1
                          1 -> 2
                  1 -> 3

choose3_ x = case x of 
                  0 -> let y = 0 ? 1
                  in case y of
                          0 -> let z = 0 ? 1
                               in case z of
                                       0 -> 0
                                       1 -> 1
                          1 -> 2
                  1 -> 3


choose3_0 x y = in case y of
                        0 -> let z = 0 ? 1
                             in case z of
                                     0 -> 0
                                     1 -> 1
                        1 -> 2

choose3_0_0 x y z = case z of
                         0 -> 0
                         1 -> 1
\end{verbatim}

Each function contains a path up to the most recent choice,
and the parameters are the local variables of the function.
Now, when we push a rule onto the backtracking stack we pick the function corresponding to the most recent
case that was evaluated.  This is similar, but not equivalent to the current position in the definitional tree.
mostly because a definitional tree can't have a local variables, and a rewrite decision can't be made on
a local variable.  they must come from the parameters.

Formally we can define this transformation in ICurry.
This allows us full generality, while keeping the amount of syntax to a minimum.
The ICurry for the origonal \texttt{choose3} function is:
\begin{verbatim}
choose3 = declare x
          x = 0 ? 1
          case x of 
               0 -> declare y
                    y = 0 ? 1
                    case y of
                         0 -> declare z
                              z = 0 ? 1
                              case z of
                                   0 -> return 0
                                   1 -> return 1
                         1 -> return 2
                    1 -> return 3
\end{verbatim}

A \textit{path} in an ICurry function that uniquely determines the location of a statment.
Since Each block in ICurry has a single statement, the only information required for a path
is the branches that have been taken.
Therefore, a path is a sequence of integers that correspond to the branches taken to reach a statement.
As an example, the statement \texttt{return 1} in the above code would have the path \texttt{[0,0,1]}.

let $paths(f)$ return the set of paths to any case statement in $f$,
and $f_p$ be the statement after traversing the body of at with path $p$,
and $vars(f,p)$ return the set of local variables declared up to $p$ in $f$.
Then we compile each function $f$ into a set of functions:
$$\textbf{f} = \bigcup_{p \in paths(e)} f_p(vars(f,p))$$\\

We can define $paths(f)$ with the following algorithm:
$$paths(case\ x\ of \{C_1 \rightarrow e_1, \\
                      C_2 \rightarrow e_2, \\
                      \ldots, \\
                      C_n \rightarrow e_n\}) \\
                      = [] \\
                      ? 1:paths(e_1) \\
                      ? 2:paths(e_2) \\
                      \ldots \\
                      ? n:paths(e_n) \\
$$\\
We can define $vars(f)$ with the following algorithm:
$$vars(delcs, _, case\ x\ of \{C_1 \rightarrow e_1, \\
                               C_2 \rightarrow e_2, \\
                               \ldots, \\
                               C_n \rightarrow e_n\}, i:path) \\
                               = decls \cup vars(e_i, path)
$$\\

That is 
