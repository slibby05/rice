\documentclass{article}
\usepackage{amsmath}
\usepackage{amsfonts}


\begin{document}
%\chapter{Compilation}

\section{The Curry Language}

\subsection{Syntax}

The syntax of Curry should look familiar to anyone with experience programming in Haskell \ref{fig:syntax}.
Functions are declared by defining equations, and new data types are declared as algebraic data types.
Function application is represented by juxtaposition, so $f\ x$ represents the function $f$ applied to the variable $x$.
Curry also allows for declaring new infix operators.
In fact, Curry really only adds two new pieces of syntax to Haskell, \textbf{fcase} and \textbf{free}.
However, the main difference between Curry and Haskell is not immediately clear from the syntax.
Curry allows for overlapping rules and free variables.
Specifically Curry is a Limited Overlapping Inductively Sequential (LOIS) Rewrite system.
Haskell, on the other hand, requires all rules to be non-overlapping.

To see the difference consider the usual definition of factorial.
\begin{verbatim}
fac :: Int -> Int
fac 0 = 1
fac n = n * fac (n-1)
\end{verbatim}

This seems like an innocuous Haskell program, however It's non-terminating for every possible input for Curry.
The reason is that \texttt{fac 0} could match either rule.
In Haskell all defining equations are ordered sequentially. which results in control flow similar to the following C
implementation.
\begin{verbatim}
int fac(int n)
{
    if(n == 0)
    {
        return 1;
    }
    else
    {
        return n * fac(n-1);
    }
}
\end{verbatim}
In fact, every rule with multiple defining equations follows this pattern
If we the equations where $p_i$ is a pattern and $E_i$ is an expression.
\begin{align*}
f\ p_1 &= E_1\\
f\ p_2 &= E_2\\
\ldots\\
f\ p_n &= E_n
\end{align*}
Then this is redefined to be semantically equivalent to the following.
\begin{align*}
f\ p_1 &= E_1\\
f\ \neg p_1 \wedge p_2 &= E_2\\
\ldots\\
f\ \neg p_1 \wedge \neg p_2 \wedge p_n &= E_n\\
\end{align*}
Here $\lnot p_i$ means that we don't match pattern $i$.
This ensures that we will ever only reduce to a single expression.
Specifically we reduce to the first expression where we match the pattern.

Curry rules, on the other hand, are unordered.
If we could match multiple patterns, such as in the case of \texttt{fac}, 
then we non-deterministically return both expressions.
Therefore \texttt{fac 0} reduces to both \texttt 1 and \texttt{fac (-1)}.
Exactly how Curry reduces an expression non-deterministically will be discussed throughout this dissertation,
but for now it is acceptable to think in terms of sets.
If the expression $e \rightarrow e_1$ and $e \rightarrow e_2$,
$e_1 \rightarrow^* v_1$ and $e_2 \rightarrow^* v_2$, then $e \rightarrow^* \{v_1, v_2\}$.
This intuition breaks down in several corner cases involving lazy evaluation,
but it works well enough for now.

The second major addition to Curry is free variables.
A free variable could have any value of the correct type,
so a free \texttt{BooL} could be either \texttt{True} or \texttt{False}.
Usually a free variable's value will be constrained during the execultion of the program.
For example.z



\begin{figure}

\begin{tabular}{rcl}
Prog     & \rightarrow & Module\textsuperscript{?} Import*  Decl*                           \\
Module   & \rightarrow & \textbf{module} Name \textbf{(} Export*  \textbf{)} \textbf{where} \\
Type     & \rightarrow & Name\ Type*                                                        \\
         & |           & TVar                                                               \\
         & |           & Type \textbf{->} Type                                              \\
         & |           & \textbf{(} (Type,)*  Type \textbf{)}                               \\
         & |           & \textbf{[} (Type,)*  Type \textbf{]}                               \\
Decl     & \rightarrow & FunDecl | DataDecl                                                 \\
DataDecl & \rightarrow & \textbf{data} Name TVar*  \textbf{=} Cons (\textbf{|} Cons*)       \\
         & |           & \textbf{type} Name TVar*  \textbf{=} Type                          \\
Cons     & \rightarrow & Name Type*                                                         \\
FunDecl  & \rightarrow & FunType\textsuperscript{?} FunDefn                                 \\
FunType  & \rightarrow & Name :: Type                                                       \\
FunDefn  & \rightarrow & Name Param* \textbf{=} Expr                                        \\
Expr     & \rightarrow & Var                                                                \\
         & |           & Lit                                                                \\
         & |           & Expr \textbf{::} Type                                              \\
         & |           & \textbf{(} Expr \textbf{)}                                         \\
         & |           & Expr Expr*                                                         \\
         & |           & Expr Op Expr                                                       \\
         & |           & \textbf{let} Binding* \textbf{in} Expr                             \\
         & |           & Expr \textbf{where} Binding*                                       \\
         & |           & \textbf{case} Expr \textbf{of} Branch*                             \\
         & |           & \textbf{fcase} Expr \textbf{of} Branch*                            \\
Binding  & \rightarrow & Name \textbf{=} Expr                                               \\
Branch   & \rightarrow & Constructor \textbf{->} Expr                                       \\
         & |           & Lit \textbf{->} Expr                                               \\
TVar     & \rightarrow & Name                                                               \\
Var      & \rightarrow & Name                                                               \\
Param    & \rightarrow & Name                                                               \\
\end{tabular}
\label{fig:Syntax}
\caption{The Syantax of Curry: Terminal symbols are in bold. Name is any valid identifier.
    Lit is any valid Integer, floating point, character, or string literal.}
\end{figure}

\begin{figure}

\begin{tabular}{rcl}
Prog        & \rightarrow & \textbf{Prog} Name Import* Data* Fun* Op*                        \\ 
Type        & \rightarrow & \textbf{TCons} QName Type*                                       \\ 
            & |           & \textbf{TVar} TVar                                               \\ 
            & |           & \textbf{FuncType} Type Type                                      \\ 
Data        & \rightarrow & \textbf{Type} QName Vis TVar* \textbf{=} Cons (\textbf{|} Cons*) \\ 
Cons        & \rightarrow & \textbf{Cons} QName Arity Vis Type*                              \\ 
Op          & \rightarrow & \textbf{Op} QName Fixity Precidence                              \\ 
Fixity      & \rightarrow & \textbf{InfixOp} | \textbf{InfixlOp} | \textbf{InfixrOp}         \\ 
Fun         & \rightarrow & \textbf{FuncDecl} QName Arity Vis Type Rule                      \\ 
Rule        & \rightarrow & \textbf{Rule} Param* Expr                                        \\ 
            & |           & \textbf{External} Name                                           \\ 
Expr        & \rightarrow & \textbf{Var} Var                                                 \\ 
            & |           & \textbf{Lit} Lit                                                 \\ 
            & |           & \textbf{Typed} Expr Type                                         \\ 
            & |           & \textbf{Comb} CombType QName Expr*                               \\ 
            & |           & \textbf{Let} Binding* \textbf{in} Expr                           \\ 
            & |           & \textbf{Case} CaseType Expr Branch*                              \\ 
Binding     & \rightarrow & \textbf{(} Name \textbf{,} Expr \textbf{)}                       \\ 
CaseType    & \rightarrow & \textbf{Rigid} | \textbf{Flex}                                   \\ 
CombType    & \rightarrow & \textbf{FuncCall} | \textbf{ConsCall}                            \\ 
            & |           & \textbf{FuncPartCall} Arity | \textbf{ConsPartCall} Arity        \\ 
Branch      & \rightarrow & \textbf{Branch} Pattern Expr                                     \\ 
Pattern     & \rightarrow & \textbf{Pattern} QName Var*                                      \\ 
            & |           & \textbf{LPattern} Lit                                            \\ 
Lit         & \rightarrow & \textbf{Intc} Int | \textbf{Charc} Char | \textbf{Floatc} Float  \\ 
Import      & \rightarrow & Name                                                             \\ 
QName       & \rightarrow & \textbf{(}Name\textbf{,} Name\textbf{)}                          \\ 
Vis         & \rightarrow & \textbf{Public} | \textbf{Private}                               \\ 
TVar        & \rightarrow & Nat                                                              \\ 
Var         & \rightarrow & Nat                                                              \\ 
Param       & \rightarrow & Nat                                                              \\ 
Arity       & \rightarrow & Nat                                                              \\ 
Precidence  & \rightarrow & Nat                                                              \\ 
\end{tabular}
\label{Syntax}
\caption{The Syantax of FlatCurry}
\caption{The Syantax of FlatCurry: Again terminal symbols are in bold, and Nam is any valid identifier.
    Nat is any valid natural number. QName is added to disambiguate names from other modules.}
\end{figure}

\subsection{Semantics}
\end{document}
