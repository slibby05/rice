\documentclass{article}
\usepackage{qtree}
\usepackage{amsmath}

\begin{document}

\section{Unboxing}
Unboxing is a useful transformation that lets us deal with primitive values, 
such as integers, characters, and floating point numbers in a more efficient manner.
In a naive functional language implementation these operations are usually much too slow.
To see what the problem is consider the following curry code:

\begin{verbatim}
1 + 2 + 3 + 4 + 5
\end{verbatim}

This seems harmless, and it should be relatively easy to evaluate.
It's just adding five integers.
However, when compiling to FlatCurry, we get the following code:

\begin{verbatim}
(((1 +$ 2) +$ 3) +$ 4) +$ 5
x +$ y = (prim_Int_plus $# v2) $# v1
\end{verbatim}

Why are there three layers of indirection?
What are \texttt{prim\_Int\_plus}, and \texttt{\$\#}?
This should be simple primitive addition.

The problem is that an \texttt{Int} in Curry isn't really a primitive type.
Because of lazy evaluation the expression \texttt{x + y} can't guarantee that
\texttt x and \texttt y will be integer values.  They may be unevaluated expressions.
Furthermore any value in Curry could be a failure.

For these reasons a Curry \texttt{Int} is really a boxed type.
That is, an \texttt{Int} is a \texttt{Node} whose tag is \texttt{Int}, and first child is the value of the integer.
This \texttt{Node} is the ``box'' for the integer.
So, every time an integer appears in an expression, we must allocate memory on the heap for the box.
This is convenient for writing the compiler, because primitive types behave like every other type,
but as we've seen it's inefficient.

\begin{verbatim}
x +$ y = (prim_Int_plus $# v2) $# v1
\end{verbatim}

The first function \texttt{+\$} is the implementation of the \texttt{+} operator for \texttt{Int} types.
It takes two Integer expressions, and adds them together.
The function \texttt{\$\#} evaluates it's argument to head normal form.
Since the head normal form for an Integer expression is an integer value itself, this evaluates the argument.
Finally \texttt{prim\_int\_plus} unboxes both of it's arguments, performs the primitive addition,
and puts the result in a new box.

The goal of the unboxing transformation is to remove code that stores and retrieves primitive values from boxes.
In the case of \texttt{1 +\$ 2}, we may be able to replace this with \texttt{prim\_Int\_plus 1 2},
but this solution won't handle more general cases.
Instead we are going to take advantage of another optimization.

\subsection{Inlining}

Inlining is crucial to getting good performance out of functional programs. \cite{inlining}
In fact it subsumes several optimizations found in traditional compilers \cite{inlining, appelComp, appelCont}.
The optimizations of constant folding, copy propagation, dead code elimination, and jump elimination are all
no longer needed.

While the concept of inlining is not very complicated, we simply replace call to a function with it's body,
it's implementation tends to be tricky. \cite{inlining}
It is helpful to distinguish a few different types of inlining.
\begin{itemize}
    \item function inlining:\\
        $f\ x = b$\\
        $... f\ e ...$\\
        $\Rightarrow$\\
        $\text{let } y = e \text{ in } ... b[x \to y] ...$\\
        We replace a function call with it's body.
        But we first bind the argument to a variable.
    \item Case cancelling:\\
        $\text{case } C x \text{ of }$\\
        $\ \ ...$\\
        $\ \ C y \rightarrow b$\\
        $\ \ ...$\\
        $\Rightarrow$\\
        $\text{let } y = x \text{ in } b$\\
        If we ever have a case applied to a constructor,
        we can completely remove the case, and replace it with the correct branch.
    \item Dead code elimination:\\
        $\text{let } y = x \text{ in } ...$\\
        $\Rightarrow$\\
        $...$\\
        If a let bound variable doesn't occur in an expression, then we can remove it entirely.
    \item Variable inlining:\\
        $\text{let } x = e_1 \text{ in } e_2 \Rightarrow e_2[x \to e_1]$
        We can replace a let bound variable with its definition.
\end{itemize}

All of these optimizations are valid in a lazy functional language.
However, this isn't true for a functional logic language.
To see the problem consider the code:

\begin{verbatim}
double x = x + x
double (0 ? 1)
\end{verbatim}

We can use function inlining to get:
\begin{verbatim}
double x = x + x
let x = (0 ? 1) in x + x
\end{verbatim}

Dead code elimination can further shorten this to:
\begin{verbatim}
let x = (0 ? 1) in x + x
\end{verbatim}

This is still valid, however is we try to apply variable inlining we get
\begin{verbatim}
let x = (0 ? 1) in (0 ? 1) + (0 ? 1)
\end{verbatim}

This is not the same expression.
We need to establish when these transformations are valid in Curry,
and furthermore when they are useful.
The problem is not entirely unique to function logic programming.
While it's be valid, it would still be a bad idea to inline:
\begin{verbatim}
let x = ack 4 3
in x + x
\end{verbatim}

First let's focus on when inlining is a valid transformation.
It turns out that the only time inlining is invalid is if 
the inlined expression is non-deterministic and has more than one parent.
If an inlined expression is deterministic, then there is clearly no problems with inlining.
If an inlined expression is non-deterministic, but has a single parent, then inlining is still valid.
Since the node has one parent, it's not shared and safe to inline.

An immediate consequence of this is that function inlining, case cancelling, and dead code elimination
are always valid.  Both function inlining and case cancelling, the inlined node only has one parent,
and dead code elimination was always valid.

This means that the only possible problem comes from variable inlining.
However, it's now easy to check if a variable can be inlined.
If the expression defining the variable is deterministic, or the variable only occurs once,
then it can be safely inlined.  This matches the results found by \cite{Inlining}

\subsection*{case canceling}
While inlining function can give some efficiency gains, it is more useful to expose other optimizations.
In particular inlining makes it much easier to apply case canceling.
Consider the following reductions

\begin{verbatim}
nonempty = not . null
=> expand
nonempty x = (not . null) x
=> inline .
nonempty x = let f = not 
                 g = null
             in f (g x)
=> inline f, g
nonempty x = let f = not 
                 g = null
             in not (null x)
=> eliminate
nonempty x = not (null x)
=> inline null
nonempty x = not (case x of
                       [] -> True
                       (_:_) -> False)
=> inline not
nonempty x = let y = (case x of
                           [] -> True
                           (_:_) -> False)
             in case y of
                     True -> False
                     False -> True
=> inline y
nonempty x = let y = (case x of
                           [] -> True
                           (_:_) -> False)
             in case (case x of
                           [] -> True
                           (_:_) -> False)
                     True -> False
                     False -> True
=> eliminate y
nonempty x = case (case x of
                        [] -> True
                        (_:_) -> False)
                  True -> False
                  False -> True
=> case-in-case
nonempty x = case x of
                  [] -> case True of
                             True -> False
                             False -> True
                  (_:_) -> case False of
                             True -> False
                             False -> True
=> Case Cancelling
nonempty x = case x of
                  [] -> False
                  (_:_) -> True
\end{verbatim}

This can take a long time, but eventually we got the code that we want for \texttt{nonempty}.
However, It's important to note that the intermediate steps doesn't really save much time.
It's not until we can actually cancel a case statement out that we save time.

\subsection*{Unboxing as Case Canceling}

Now that we have a mechanism for inlining and case canceling,
it would be nice to apply it to unboxing.
This idea was proposed by Jones and Launchbury \cite{unboxing},
and is pretty standard in lazy functional languages.

As discussed previously the problem with unboxing is that boxed values are entirely implicit.
The goal is to make the box explicit, and use case canceling to eliminate it.
We achieve this by making a new type
\begin{verbatim}
data Int = Int Int#
\end{verbatim}
New \texttt{Int} is a boxed integer and \texttt{Int\#} is an unboxed integer.
So to represent the integer literal 4 as \texttt{Int 4}.
If we want to pattern match an integer you pattern match the integer constructor first.
\begin{verbatim}
case x of
     Int x# -> ...
\end{verbatim}
Here, the \texttt x must be of the form \texttt{Int x\#}, and \texttt{x\#} must be a literal integer.
We use \texttt{\#} to distinguish variables that contain primitive, unboxed values.

We can rewrite the integer addition from before using explicit case statements.
\begin{verbatim}
x +$ y = case x of
              Int x# -> case y of
                             Int y# -> case x# +# y# of
                                            z# -> Int z#
\end{verbatim}

Now the variables \texttt x and \texttt y are boxed integers,
and the variables \texttt{x\#} and \texttt{y\#} are unboxed integers.
We also add a case to evaluate the primitive operation.
This case can usually be remove through case cancellation.
Aside from possible optimizations, this representation is useful for code generation.
We don't have to worry about evaluating \texttt x or \texttt y to normal form.
That's enforced by the case itself.
The function \texttt{prim\_int\_plus} can now be replaced with normal integer addition in the generated C code.

This works for primitive addition, but what about pattern matching.
Well, literal patterns are handled in exactly the same way.
Suppose we have the function:
\begin{verbatim}
f x = case x of
           1 -> True
           2 -> False
           3 -> True
\end{verbatim}

This function is replaced by 
\begin{verbatim}
f x = case x of
           Int x# -> case x# of
                          1 -> True
                          2 -> False
                          3 -> True
\end{verbatim}

This transformation is easy to implement, and again helps in generating code.
We've just made this explicit at the FlatCurry stage.

So we have a transformation that works, and simplifies generating code, but how does it help with unboxing.
Lets consider our original expression we were trying to optimize.


\begin{verbatim}
1 + 2 + 3 + 4 + 5
\end{verbatim}
the boxed version of this program is
\begin{verbatim}
(Int 1) + (Int 2) + (Int 3) + (Int 4) + (Int 5)
\end{verbatim}
after applying inlining we end up with
\begin{verbatim}
case (Int 1) of
        (Int x1) -> case (Int 2) of
           (Int x2) -> case (Int 3) of
              (Int x3) -> case (Int 4) of
                 (Int x4) -> case (Int 5) of
                    (Int x5) -> case (prim_int_add
                                      (prim_int_add
                                       (prim_int_add
                                        (prim_int_add x1 x2)
                                        x3)
                                       x4)
                                      x5) of
                                      x# -> Int x#
\end{verbatim}

Now the case canceling becomes obvious,
and we end up with:
\begin{verbatim}
Int (prim_int_add 1 (prim_int_add 2 (prim_int_add 3 (prim_int_add 4 5))))
case (prim_int_add (prim_int_add (prim_int_add (prim_int_add 1 2) 3) 4) 5) of
      x# -> Int x#
\end{verbatim}

Even better, at this point we can apply constant folding and get:
\begin{verbatim}
case 15 of
     x# -> Int x#
\end{verbatim}
which further reduces to:
\begin{verbatim}
Int 15
\end{verbatim}

\subsection*{Unboxing functions}

Unboxing as we have defined it now works well for expressions.
However, we would also like to be able to unbox function calls too.
This can be done with a simple transformation.

I'll use the factorial function as an example.
\begin{verbatim}
fac :: Int -> Int
fac x = if x == 0 then 1 else x * fac (x-1)
\end{verbatim}

While compiling to FlatCurry \texttt{if \_ then \_ else \_} is replace by the function:
\begin{verbatim}
if_then_else True  t f = t
if_then_else False t f = f
\end{verbatim}

We can already improve this by applying the inlining and unboxing.
I will use \texttt{==\#, -\#}, and \texttt{*\#} for the unboxed
integer equality, subtraction, and multiplication operators respectively.
\begin{verbatim}
fac :: Int -> Int
fac x = case x of
         Int x# -> case x ==# 0  of
                    True  -> (Int 1) 
                    False -> case x# -# 1 of
                              x1# -> case fac (Int x1#) of
                                      Int facX# -> case (x# *# facX#) of
                                                    r# -> Int r#
\end{verbatim}
Now, if \texttt{fac} we to take and return a primitive type, then
we could remove all of the unboxing code in this funciton.
So, we make a new function \texttt{fac\#}
\begin{verbatim}
fac# :: Int# -> Int#
fac# x# = case x ==# 0 of
               True  -> 1 
               False -> case x# -# 1 of
                             x1# -> case fac# x1# of
                                         faxX# -> case x# *# facX# of
                                                       r# -> r#

fac :: Int -> Int
fac x = case x of 
             Int x# -> case (fac# x#) of
                            facX# -> Int facX#
\end{verbatim}

\end{document}

