
basic theory of inlining

case cancelling
dead code elimination
beta reduction
 - ordering functions
 - the call graph
 - loop breakers

If we have a non-trivial strongly connected component,
then we have possible recursion in our function calls.
While we want recursion in our language, we absolutely cannot
deal with recursion while inlining.
For a simple example, suppose we have the functios.

\begIn{verbatim}
f x = g x
g x = f x
\end{verbatim}

If we were to decide to $\beta$-reduce the expression \texttt{f 5}
we could end up in an infinite loop trying to simplify the expression.
While this doesn't hurt the generated code, it's important that our inlining algoirthm terminate.

For an even more insidious example, we could has an expanding expression.
\begin{verbatim}
f x = g (x+1)
g x = f (x+1)
\end{verbatim}

Now when we $\beta$-reduce \texttt{f 5} our expression increases in size.
\begin{verbatim}
f 5
-> g (5 + 1)
-> f ((5 + 1) + 1)
-> g (((5 + 1) + 1) + 1)
...
\end{verbatim}

While an expression increasing in size might be warrented in some cases, 
we clearly need to avoid this problem with recursive function calls.
Therefore it's very important that we can detect possible recursion.
Fortunately we already have the tool for this.
Consider the call graph for \texttt f and \texttt g.

\begin{graph}
f -> g
g -> f
\end{graph}

The two vertices form a cycle.
In fact this generalizes to any collection of function calls.
Any non-trivial strongly connected component of a call graph will contain recursion.
If we were to eliminate all of these components from our call graph, 
then we wouldn't need to worry about recursion durring the inlining phase.

The trick to this is actually pretty simple.
We just pick a function in a strongly-connected component, and remove it from consideration.
We call these functions loopbreakers after \cite{inlining}.
Once a function has been designated as a loop breaker, it's never allowed to be inlined.
We then remove every edge to the loopbreaker from the call graph.
This process repreats until our call graph is acyclic.
As a side effect, when we topologically sort our graph we have a nice ordering for optimiing functions.
We are gaurenteed any 


 - function symbol table


inlining
 - variable dependancy graph
 - loop breakers again
 - size of a variable
 - work of a variable

